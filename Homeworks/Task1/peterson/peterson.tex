\documentclass{article}
\usepackage[T2A]{fontenc}
\usepackage[english,russian]{babel} 
\usepackage{cmap} 
\usepackage{soulutf8}
\usepackage[utf8]{inputenc}
\usepackage{amsmath}

\begin{document}
\subparagraph{\textit{ Задача ''Алгоритм Петерсона [peterson]'', Ткаченко Дмитрий, команда PQ.}}

\subparagraph{Решение:} Приведем фрагмент кода и последовательность выполнения инструкций на процессоре, доказывающую, что при замене мест операций в doorway-секции, алгоритм Петерсона перестает гарантировать принцип взаимного исключения:
\newline
void lock(int t) \{
\newline
//1	victim.store(t);
\newline
//2 want[t].store(true);
\newline
	while (want[1-t].load() && victim.load() == t) \{
	\newline
		//wait
		\newline
	\}
	\newline
\} 
\newline 
Здесь комментарии 1 и 2 будут означать выполнение данной строчки кода. Инструкцию процессора зададим видом n-k, где n - номер потока, который будет выполнять команду k (это 1, 2 и w - попытка захода в wait-секцию, то есть проверка условий цикла while).

Приведем же наконец порядок выполнения инструкций на процесосре:
\newline
\subparagraph{1)} 0 - 1 (want[0] == false, want[1] == false, victim == 0)
\subparagraph{2)} 1 - 1 (want[0] == false, want[1] == false, victim == 1)
\subparagraph{3)} 1 - 2 (want[0] == false, want[1] == true, victim == 1)
\subparagraph{4)} 1 - w (want[0] == false, want[1] == true, victim == 1). Попытка захода в wait-секцию, условия цикла while не выполнены (want[0] = false) и поток 1 заходит в критическую секцию.
\subparagraph{5)} 0 - 2 (want[0] == true, want[1] == true, victim == 1)
\subparagraph{6)} 0 - w (want[0] == true, want[1] == true, victim == 1).
Попытка захода в wait-секцию, условия цикла while не выполнены (victim == 1) и поток 0 заходит в критическую секцию.
\newline
\newline
\subparagrapg{\textbf{Принцип взаимного исключения нарушен.}}
\subparagraph{Ответ:} Нет, не будет.

\end{document}

